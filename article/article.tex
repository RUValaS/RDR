\documentclass[titlepage]{article}
\usepackage[upright]{fourier}
\usepackage[utf8]{inputenc}
\usepackage[english]{babel}
\usepackage[margin=1in]{geometry}
\usepackage[T1]{fontenc}

\usepackage[colorlinks=true,linkcolor=teal]{hyperref}
\usepackage{amssymb,amsmath,minted,float,graphicx,textcomp,systeme,listings,physics,mathtools,ifthen,esvect,hyperref,fancyhdr,lastpage,marginnote,chngcntr,fancyvrb,subcaption,svg}

\usepackage[backend=biber,style=numeric]{biblatex}
\addbibresource{article.bib}

\DeclareMathOperator{\e}{e}
\def\mathbi#1{\textbf{\em #1}}

\setlength\parindent{0pt}

%%%%%%%%%%%% PARAMETRES
\newcommand{\UE}{4XX}
\newcommand{\type}{}%TD-TP-COURS
\newcommand{\nb}{0}%nb 
\newcommand{\sbt}{}%soustitre
\author{Gabriel ROBERT-DAUTUN}
\date{Mars 2022}

% mise en page (header, compteur, fancy)
\pagestyle{fancy}
\lhead{Dynamic image reconstruction}
\rfoot{\thepage /\pageref{LastPage}}
\lfoot{}
\cfoot{}
\renewcommand{\footrulewidth}{0.4pt}

%reset compteur section par partie
\counterwithin*{section}{part}

% permet notes + simples en marge à gauche (eg compteur de question)
\reversemarginpar
\newcommand{\mgn}[1]{\marginnote{#1}}

%sinon warning mdr
\setlength{\headheight}{13.07225pt}

\newcommand{\hinv}[1]{#1^{\circ-1}} % inverse de hadamard
\newcommand{\fnorm}[1]{|\vert#1|\vert_{F}} % norme de Frobenius
\newcommand{\moy}[1]{\boldsymbol{\mathsf{m}}_{#1}}
\newcommand{\autocorr}[1]{\expval{(#1)(#1)^H}}
\newcommand{\ccorr}[2]{\expval{(#1)(#2)^H}}

\renewcommand{\expval}[1]{\text{E}\left[#1\right]}
\newcommand{\w}{\boldsymbol{w}}
\renewcommand{\v}{\boldsymbol{v}}
\newcommand{\Q}{\boldsymbol{\mathsf{Q}}}
\newcommand{\R}{\boldsymbol{\mathsf{R}}}
\renewcommand{\H}{\boldsymbol{\mathsf{H}}}
\newcommand{\A}{\boldsymbol{\mathsf{A}}}

\newcommand{\x}{\boldsymbol{x}}
\newcommand{\y}{\boldsymbol{y}}

\newcommand{\xp}{\widehat{\x}_{k|k-1}}
\newcommand{\xa}{\widehat{\x}_{k-1|k-1}}
\newcommand{\xe}{\widehat{\x}_{k|k}}
\newcommand{\yp}{\widehat{\y}_{k|k-1}}

\newcommand{\Pp}{\boldsymbol{\mathsf{P}}_{k|k-1}}
\newcommand{\Pa}{\boldsymbol{\mathsf{P}}_{k-1|k-1}}
\newcommand{\Pe}{\boldsymbol{\mathsf{P}}_{k|k}}
\renewcommand{\P}{\boldsymbol{\mathsf{P}}}

\title{%
	Radioastronomical image reconstruction using Kalman filter under ionospheric perturbations}

\begin{document}
	
	\begin{titlepage}
	
	\vspace*{.3\textheight}
	\huge
	\centering
	\textbf{Dynamic radioastronomic image reconstruction using Kalman filter under ionospheric perturbations}
	
	\vspace{1cm}
	\LARGE
	G. Robert-Dautun
	
	\vfill
	\large
	Supervised by I. Vin
	
	\vspace{0.8cm}
	
	\Large
	Satie\\
	France\\
	July 2022
	
	\end{titlepage}
	
	\newpage
	\tableofcontents
	
	\newpage
	
	\part{Introduction}
	
		A Kalman filter is a linear quadratic estimator, using observations, state-transition model and knowledge about noise and errors to compute a more accurate result than observations alone. Radio astronomical measures being prone to noise and errors, usage of this filtering method seems \emph{a priori} a good way to improve accuracy of reconstruction. \\
		One objective of this paper is to evaluate if measure conditions and results are compatible with the filtering.
	
	
	
	\newpage
	\part{Notations and problem}
	\section{Notations}
	
	\subsection{General notations}
	
	$\expval{\bullet}$ denotes the expected value of a random variable/vector/matrix.
	
	\paragraph{}Bold symbols represent vectors, normal ones represent scalars
	
	
	
		\subsection{Hadamard product}
	
	%On introduit la notation $\odot$ associée au produit de Hadamard, qui effectue la multiplication élément à élément de deux matrices de même taille :
	Hadamard product, denoted $\odot$, is the element-wise product of same sized matrices :
	
	$$
	\forall (A,B)\in\left(\mathbb{C}^{m\times n}\right)^2, \quad A\odot B\in\mathbb{C}^{m\times n} \quad\text{and}\quad \forall (i,j)\in\llbracket1;m\rrbracket\times\llbracket1;n\rrbracket,\, (A\odot B)_{i,j} = A_{i,j}\times B_{i,j}
	$$
	
	%On introduit également l'inverse de Hadamard, qui inverse élément par élément une matrice
	We can further introduce Hadamard inversion :
	$$
	\forall A \in \left(\mathbb{C}\backslash\{0\}\right)^{m\times n},\quad \hinv{A}\in\left(\mathbb{C}\backslash\{0\}\right)^{m\times n} \quad\text{and}\quad \forall (i,j)\in\llbracket1;m\rrbracket\times\llbracket1;n\rrbracket,\, \left(\hinv{A}\right)_{i,j} = \left(A_{i,j}\right)^{-1}
	$$ 
	
%	Ainsi que la notation $\oslash$ désignant la division de Hadamard :
	And notation $\oslash$ being Hadamard division :
	$$
	\forall (A,B)\in\mathbb{C}^{m\times n}\times\left(\mathbb{C}\backslash0\right)^{m\times n}, \quad A\oslash B\in\mathbb{C}^{m\times n} \quad\text{and}\quad \forall (i,j)\in\llbracket1;m\rrbracket\times\llbracket1;n\rrbracket,\, (A\oslash B)_{i,j} = \frac{A_{i,j}}{B_{i,j}}
	$$
	
	\subsection{Norms and distances}
	
	%On dénote $\fnorm{\bullet}$ la norme de Frobenius :
	In order to compute distances between matrices, we note $\fnorm{\bullet}$ Frobenius norm :
	$$
	A\in\mathcal{M}_{m,n}(\mathbb{K}) \quad \fnorm{A} := \sqrt{\tr{AA^H}} = \sqrt{
		\;\;\;\sum_{
			\mathclap{
				\substack{
					1\le i\le m \\1\le j\le n
				}
			}
		}\abs{A_{ij}}^2}
	$$
	
	%Qui induit sur $\mathcal{M}_{m,n}(\mathbb{D})$ la distance normalisée $d_1$ :
	Inducing the distance $d_1$ :
	$$
	\forall(A,B)\in\left(\mathcal{M}_{m,n}(\mathbb{K})\right)^2\quad d_1(A,B) = \frac{\fnorm{A-B}}{mn}
	$$
	
	%Ainsi que la distance $d_2$ :
	In order to measure influence of a specific element, the ??? distance $d_2$ is the following :
	$$
	\forall(A,B)\in\left(\mathcal{M}_{m,n}(\mathbb{K})\right)^2\quad d_2(A,B) = \sum_{
		\mathclap{
			\substack{
				1\le i\le m \\1\le j\le n
			}
		}
	}\abs{A_{ij} - B_{i,j}}
	$$
	
	
	\section{Evolution model and Kalman filtering}
	
	We consider a set $\{x_k\}$ of $n$ vector as theoretical images taken at a regular time interval. Those images are measured by an antenna mesh which UV plane will be written $H$, more generally written as observation model. Measurements vectors will be designated by $n$ measures $\{y_k\}$.\\
	We will consider the following model to describe time evolution and observation equation :
	$$
		\begin{cases}
			\x_k = A\x_{k-1} + \w_{k-1}\\
			\y_k = H\x_k + \v_k
		\end{cases}
	$$
	
	Where :
	\begin{itemize}
		\item $A$ is the linear state-transition model between time $k-1$ and $k$, assuming evolution is time invariant
		\item $\{\w_k\}$ is the state noise sequence
		\item $\{\v_k\}$ is the measurements noise sequence
	\end{itemize}
	
		
	In order to build images dynamically with better precision, a Kalman filter has been implemented.\\
	
	This filter requires more inputs to estimate the state of the system :
	\begin{itemize}
		\item $\{\Q_k\}$, the covariance of the process noise sequence
		\item $\{\R_k\}$, the covariance of the observation noise sequence
	\end{itemize}

	Which can be rewritten as
	$$
		\begin{cases}
			\Q_k = \expval{\w_{k-1}\w_{k-1}^H} - \expval{\w_{k-1}}\expval{\w_{k-1}}^H \\
			\R_k = \expval{\v_{k}\v_{k}^H} - \expval{\v_k}\expval{\v_k}^H 
		\end{cases}
	$$
	
	Those inputs are not \emph{a priori} known, but it has been shown that using the \emph{autocovariance least-squares} (ALS) method provides a good estimate of those matrices. [CITATION NEEDED] \\
	We will assume that these inputs are known perfectly except when specified otherwise.\\
	
	Kalman filtering works in 2 steps :
	\paragraph{Prediction} : This step predicts the state of the system at next time-step through de transition-state model, and the predicted measurement :
	$$
	\begin{cases}
		\xp = A\xa + \moy{\w_k}\\
		\yp = H\xp + \moy{\v_k}
	\end{cases}
	$$
	And the covariance matrix :
	$$
		\Pp = \autocorr{\xp - \x_k} = H\Pa H^H + \Q_k
	$$
	This result can be computed using least-square estimation of the prediction error covariance\cite{intro_KF}
	
	\paragraph{Update} : Using prior knowledge about noise and observation model, the main goal is to build an image from both prediction and measurement using maximum likelihood estimation in order to enhance the resulting image. This can be written as looking for $K_k$, known as \emph{Kalman gain}, such as :
	\begin{equation}
		\xe = \xp + K_k\left(\y_k - \yp\right)
	\end{equation}
	
	minimises the quadratic error :
	\begin{equation}
		K_k = \arg\min_K \Pe = \arg\min_K \autocorr{\xe - \x_k}
	\end{equation}
	
	The resulting expression for $K_k$ is \cite{intro_KF} :
	\begin{equation}
		K_k = \Pp H^H\left(H\Pp H^H + \R_k\right)^{-1}
	\end{equation}

	Thus the update phase can be written :
	\begin{equation}
		\begin{cases}
			\xe = \xp + K_k\left(\y_k - \yp\right) \\
			\Pe = \left(\mathbb{I} - K_kH\right)\Pp
		\end{cases}
	\end{equation}
	
	
	\newpage
	\part{Reliability of Kalman Filter under perturbations}
	\section{Perturbations sources}
	
	In radio astronomy, many factors can and will affect measures. Those can be :
	\paragraph{Initialisation error} : Kalman filtering requires initializing the first image and the first correlation matrix. Those cannot be exact and will present errors. \\
	Concerning the initial image, multiple precise and efficient methods exist \cite{bible}, and from this initial image the initial correlation matrix can ben initialised\cite{intro_KF} with precision as 
	$$
		\P_0 = \autocorr{\x_0} - \moy{\x_0}\moy{\x_0}^H
	$$
	
	\paragraph{Error covariance matrices} : as mentioned above, these matrices set can be estimated with good accuracy using ALS algorithms [CITATION NEEDED]. It is to be noted that errors on these matrices will not induce new errors in the estimated images, as they are only used to compute the Kalman gain. This implies that inaccurate error matrices can only spread already existing errors on estimated images.
	
	\paragraph{Antennas informations} : Multiple equations in Kalman filtering relies on the observation model $\H$, which is computed as the UV plane FFT covered by the antenna mesh. The limited knowledge of this mesh can create errors in $\H$, particularly when antennas positions and directions cannot be perfectly measured.
	
	\paragraph{Ionospheric perturbations} : Radio astronomical observations can be affected by the passage of the observed electromagnetic waves through the ionosphere \cite{iono}. This can affect the observation model $\H$ as it modifies the perceived direction of the electromagnetic wave measured, and calibration or correction must be made to compensate these perturbations.
	
	\section{Effects of perturbations}
	
	These perturbations create errors on the matrices used in reconstruction and prediction with Kalman filtering. The goal of this part is to evaluate how much these error affect prediction and reconstruction, to determine if Kalman filtering can be realistically used in radio-astronomic context. \\
	
	\subsection{Methodology}

	In order to measure precisely the effects, the following methodology will be used :
	\begin{itemize}
		\item 
	\end{itemize}
	
	\subsection{Position}
	\subsection{Direction}
	
	\newpage
	\part{Experimental correction}
	\section{implementation}
	\section{Measures}
	
	
	\newpage
	\printbibliography
\end{document}

\begin{figure}[H]
	\centering
	\includegraphics[width=0.7\linewidth]{}
	\caption{}
\end{figure}

\begin{figure}[H]
	\centering
	\begin{subfigure}{.5\textwidth}
		\centering
		\includegraphics[width=.4\linewidth]{image1}
		\caption{}
	\end{subfigure}%
	\begin{subfigure}{.5\textwidth}
		\centering
		\includegraphics[width=.4\linewidth]{image2}
		\caption{}
	\end{subfigure}
	\caption{}
\end{figure}
