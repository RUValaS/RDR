\documentclass[titlepage]{article}
\usepackage[upright]{fourier}
\usepackage[utf8]{inputenc}
\usepackage[english]{babel}
\usepackage[margin=1in]{geometry}
\usepackage[T1]{fontenc}

\usepackage[colorlinks=true,linkcolor=teal]{hyperref}
\usepackage{amssymb,amsmath,minted,float,graphicx,textcomp,systeme,listings,physics,mathtools,ifthen,esvect,hyperref,fancyhdr,lastpage,marginnote,chngcntr,fancyvrb,subcaption,svg}

\DeclareMathOperator{\e}{e}
\def\mathbi#1{\textbf{\em #1}}

\setlength\parindent{0pt}

%%%%%%%%%%%% PARAMETRES
\newcommand{\UE}{4XX}
\newcommand{\type}{}%TD-TP-COURS
\newcommand{\nb}{0}%nb 
\newcommand{\sbt}{}%soustitre
\author{Gabriel ROBERT-DAUTUN}
\date{Mars 2022}

% mise en page (header, compteur, fancy)
\pagestyle{fancy}
\lhead{\UE\, - \type\ifthenelse{\nb > 0}{\nb}{}}
\rfoot{\thepage /\pageref{LastPage}}
\lfoot{}
\cfoot{}
\renewcommand{\footrulewidth}{0.4pt}

%reset compteur section par partie
\counterwithin*{section}{part}

% permet notes + simples en marge à gauche (eg compteur de question)
\reversemarginpar
\newcommand{\mgn}[1]{\marginnote{#1}}

%sinon warning mdr
\setlength{\headheight}{13.07225pt}

%compteur questions
\newcounter{question}
\setcounter{question}{1}
\newcounter{subq}
\setcounter{subq}{1}

\newcommand{\rsubq}{\setcounter{subq}{1}}
\newcommand{\question}{\mgn{\thequestion .}\stepcounter{question}\rsubq}
\newcommand{\rstq}{\setcounter{question}{1}\rsubq}
\newcommand{\skipq}[1]{\addtocounter{question}{#1}\rsubq}
\newcommand{\subq}{\noindent\alph{subq})\stepcounter{subq}} %a modifier pour mettre dans le margin

\title{%
	Radioastronomical image reconstruction using Kalman filter under ionospheric perturbations}

\begin{document}
	
	\begin{titlepage}
	
	\huge
	\textbf{Reconstruction dynamique d'images radioastronomiques de trous noirs}
	
	\vspace{1.5cm}
	\LARGE
	\textbf{Gabriel Robert-Dautun}
	
	\vfill
	\large
	Encadré par Mme Vin
	
	\vspace{0.8cm}
	
	\includegraphics[width=0.75\textwidth]{logo.PNG}
	
	\Large
	Satie
	France\\
	July 2022
	
	\end{titlepage}
	
	\newpage
	\tableofcontents
	
	\newpage
	
	
\end{document}

\begin{figure}[H]
	\centering
	\includegraphics[width=0.7\linewidth]{}
	\caption{}
\end{figure}

\begin{figure}[H]
	\centering
	\begin{subfigure}{.5\textwidth}
		\centering
		\includegraphics[width=.4\linewidth]{image1}
		\caption{}
	\end{subfigure}%
	\begin{subfigure}{.5\textwidth}
		\centering
		\includegraphics[width=.4\linewidth]{image2}
		\caption{}
	\end{subfigure}
	\caption{}
\end{figure}
